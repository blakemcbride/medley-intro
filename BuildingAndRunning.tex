@c -*-texinfo-*-

@chapter Building & Running

Medley is open-source and may be obtained from GitHub.  It is
currently portable to Linux and Apple Mac systems.
The core is written in portable C but the system depends on the
X11 system for its display.

@section Building The System

@subsection Obtaining The System

The Medley system comes in two parts.  The first is the C-based
virtual machine named ``maiko''.  It may be obtained from
@uref{https://github.com/Interlisp/maiko}.

The remainder of the system may be obtained from
@uref{https://github.com/Interlisp/medley}.

Once the system has been cloned, you should have two directories named
@code{maiko} and {medley}.

@subsection Linux

On Linux, the system may be built with the following commands:

@example
$ cd maiko/bin
@end example

At this point you will need to edit the file named
@code{makefile-linux.x86_64-x} if you are using GCC rather than CLANG.
Comment out the line (with a #) that defines @code{CC} for
@code{clang} and uncomment the line (delete the #) for the line that
defines @code{CC} for @code{gcc}.

@example
$ ./makeright x
@end example

This creates the virtual machine in the @code{maiko/linux.x86_64} directory.

@subsection Mac

Building for the Mac requires an X-Server.  This can be freely obtained 
at @uref{https://www.xquartz.org/}.  Download and install it.

After that, Medley build just like it does on Linux:

@example
$ cd maiko/bin
$ ./makeright x
@end example



@section Running The System

The system can be run by typing:

@example
$ cd medley
$ ./run-medley
@end example

Or, if you wish to start Medley up with a particular image, use:

@example
$ cd medley
$ ./run-medley <image-file-name>
@end example

Once the system comes up, give it a few seconds to initialize.

The first time the system is run it loads the system image that comes
with the system.  When you exit the system the state of your machine is saved
in a file named @code{~/lisp.virtualmem}.  Subsequent system startups 
load the @code{~/lisp.virtualmem} image by default.

@section Exiting The System

The system may be exited from the Interlisp prompt by typing:

@example
(LOGOUT)
@end example

Or from the Common Lisp prompt with:

@example
(IL:LOGOUT)
@end example

When you logout of the system, Medley automatically creates a binary
dump of your system located in your home directory named
``lisp.virtualmem''. The next time you run the system, if you don't
specify a specific image to run, Medley restores that image so that
you can continue right where you left off.


