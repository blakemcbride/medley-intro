@c -*-texinfo-*-

@chapter Architecture

The Medley system operates in a virtual machine.  Lisp code is either
interpreted or compiled into (essentially) byte-code for this virtual
machine.  Use of a virtual machine in this way, while more popular
today with environments such as Java and C#, was quite a bit more
unique and innovative when Medley was developed.  As it does with
other languages, use of a virtual machine made Medley portable.

The Medley system is comprised of two major pieces.  The first part is
the virtual machine.  This is called ``maiko''.  The remainder of the
system is written in a combination of Interlisp and Common Lisp.

The Medley system is image-based (like Smalltalk) rather than
file-based (like most other systems such as C, Java, C#, Python,
etc.).  In file-based systems, source code resides in disk files.
Interpreters interpret this code and compilers convert these files
into machine code (placed in other files) for execution.  Development
amounts to the editing of these source files.

In an image-based systems you are working on a live, running system.
As code is written, the running system evolves.  At various points,
the user can save an ``image'' of that running system.  What that is
is a single, binary, copy of the entire running system.  It basically
saves the state of the running machine at the point the image is
saved.  Later, the system may be restarted with a particular image
and the system resumes from the exact point from which it was saved.

With an image-based system you can still write your source code to
disk files for various purposes but that is not the principal mode of
development.  One of the beauties of image-based development is that
when you want to see or change something, the entire system is at your
immediate disposal.  You do not need to search for the source files.
Also, changes that you make take effect immediately.  There is no need
to rebuild and redeploy the system.

As opposed to having two language sub-systems (Interlisp and Common
Lisp), Medley integrates the two such that both or either can be used.
Interlisp functions as the ``IL'' package of Common Lisp.


